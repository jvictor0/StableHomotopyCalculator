\section{The Adams Spectral Sequence and Differentials}

I will do section 2 first, if that is ok with you.

\subsection{Extended Powers of Cells}

An element of $E_r^{s,n+s}(X,Y)$ can be represented as a map
\[(CX,X)\to (Y_s,Y_{s+r}\]
and of course the cone on a sphere is a disk.  

Consider the space $S^{n-1}$.  We have three spaces we care about, which we will give stupid names
\[\Gamma_1=\Sigma S^{2n-2}=S^{2n-1}\]
\[\Gamma_0=C\Gamma_1=D^{2n}\]
\[\Gamma_2=S^{2n-2}\]
and of course we have inclusion maps given be the standard cell structure
\[\Gamma_2\to\Gamma_1\to\Gamma_0\]

Let $W^k$ be the $k$-skeleton of $S^\infty=E\Z/2$.  
Then we have
\[W^k\ltimes_{\Z/2} \Gamma_0 \cong C(W^k\ltimes_{\Z/2} \Gamma_1)\]
\[\Sigma (W^k\ltimes_{\Z/2} \Gamma_1)\cong W^k\ltimes_{\Z/2} S^{2n}\]


\newcommand{\extpow}[0]{\ltimes}
From now on, let $\extpow = \ltimes_{\Z/2}$

since for any map $f:A\to X$ the cofiber is $CA\cup_f X$, we have that
\[W^{k-1}\extpow \Gamma_1\to W^k\extpow \Gamma_1\to W^{k-1}\extpow \Gamma_0\cup W^k\extpow\Gamma_1\]
is a cofibration



Recall/observe
\[W^k/(\Z/2) \cong P^k\]
\[\frac{W^k\extpow \Gamma_i}{\W^{k-1}\extpow\Gamma_i}=\Sigma^k\Gamma_i\]

Thus by the previous chapter we have
\[W^k\extpow \Gamma_2=\Sigma^{n-1}P^{n-1+k}_{n-1}\]
\[W^k\extpow \Gamma_1=\Sigma^{n-1}P^{n+k}_n\]

Since the cofiber of 
\[\Sigma^{2n-1} P^{2n-2+k}_{2n-2}\to \Sigma^{2n-1} P^{2n-1+k}_{2n-2}\]
is a sphere, by above we have
\[W^{k-1}\extpow \Gamma_0\cup W^k\extpow\Gamma_1=\Sigma^{n-1}P^{n+k}_{n+k-1}=S^{2n+k-1}\]


Now we have a 3-d diagram.  

\begin{diagram}
  S^{2n+k-2} & & \rTo & & e^{2n+k-1} & & \\
  & \rdTo_a & & & \vLine & \rdTo_{a_1} & \\
  \dTo & & W^{k-1}\ltimes \Gamma_1 & \rTo & \HonV & & W^{k}\ltimes\Gamma_1 \\
  & & \dTo & & \dTo & & \\
  e^{2n+k-1} & \hLine & \VonH & \rTo & S^{2n+k-1} & & \dTo \\
  & \rdTo_{a_2} & & & & \rdTo^{a_3} & \\
  & & W^{k-1}\ltimes \Gamma_0 & & \rTo & & W^k\ltimes \Gamma_1\cup W^{k-1}\ltimes\Gamma_0 
\end{diagram}

where $a$ is the attaching map of the top cell, $a_1$ is an inclusion, $a_2=a\wedge 1$ and $a_3$ is an equivalence.  


\begin{Lemma}
given any map 
\[f:(W^k\ltimes \Gamma_1\cup W^{k-1}\ltimes\Gamma_0,W^{k-1}\ltimes\Gamma_1)\to (Y,B)\]
if $\pi:Y\to Y/B$, then, with appropriate choice of sign, 
\[\pi fa_3=fa_1-fa_2\]
\end{Lemma}


Let $v=v_2(n+k)$ and $a\in \pi_{2n+k-2}(W^{k-1}\ltimes\Gamma_1)$.  
$v$ is, by deffinition, the largest number such that $a$ factors through $W^{k-v}\ltimes\Gamma_1$. 
Our commutative cube can now become 


\begin{diagram}
S^{2n+k-2} & & \rTo & & e^{2n+k-1} & & \\
& \rdTo_a & & & \vLine & \rdTo_{a_1} & \\
\dTo & & W^{k-v}\ltimes \Gamma_1 & \rTo & \HonV & & W^{k-v}\ltimes\Gamma_1\cup e^{2n+k-1} \\
& & \dTo & & \dTo & & \\
e^{2n+k-1} & \hLine & \VonH & \rTo & S^{2n+k-1} & & \dTo \\
& \rdTo_{a_2} & & & & \rdTo^{a_3} & \\
& & W^{k-v}\ltimes \Gamma_0 & & \rTo & & W^{k-v}\ltimes\Gamma_0\cup e^{2n+k-1}
\end{diagram}

the lemma is modified accordingly: $f$ need only take $W^{k-1}\ltimes\Gamma_1$ to $B$ and the term $fa_2$ factors through $\W^{k-v}\ltimes\Gamma_0$.  

\subsection{Chain Level Calculations}

We give $\Gamma_0$ a very particular cell structure.  Since $\Gamma_0=D^{2n}=D^n\wedge D^n$, we can let $D^n=S^{n-1}\cup_1 D^{n}$.  We call the chain from the $n$-cell $x$ and the boundry $dx$, so that
\[C_*(\Gamma_0)=\langle x,dx\rangle\otimes\langle x,dx\rangle\]
We have then that
\[C_*(\Gamma_1)=\langle x\otimes dx,dx\otimes x, dx\otimes dx\rangle\]
and 
\[C_*(\Gamma_2)=\langle dx\otimes dx\rangle\]

We can then of course write
\[C_*(W^k\ltimes_{\Z/2})=\W(k)\otimes_{\Z/2}C_*\Gamma_i\]
and let $s$ be the obvious contracting homotopy.  
The chain differentials are easy to calculate, so heres what they are
\[d(dx\otimes dx)=0\]
\[d(x\otimes dx)=dx\otimes dx = d(dx\otimes x)\]
\[d(x\otimes x)=dx\otimes x+(-1)^nx\otimes dx\]

\begin{Lemma}
  in $C_*(W^{i+1}\ltimes\Gamma_1)$, we have, if $i\ne n$ mod 2,
  \[e_{i+1}\otimes dx\otimes dx\sim (-1)^n e_i\otimes d(x\otimes x)\]
  and if $i=n$ mod 2, we have
  \[e_{i+1}\otimes dx\otimes dx\sim (-1)^n e_i\otimes d(x\otimes x) + 2e_i\otimes x\otimes dx\]
\end{Lemma}

\begin{proof}
  Simply calculate $d(e_{i+1}\otimes x\otimes dx)$ and you get the relation you need.
\end{proof}

\begin{Lemma}
  The inclusion
  \[W^{i+1}\ltimes \Gamma_{j+1}\to W^{i+1}\ltimes\Gamma_j\]
  is homotopic to a map
  \[e:W^{i+1}\ltimes \Gamma_{j+1}\to W^{i}\ltimes\Gamma_j\]
  If $j=1$, then the map in mod-2 homology is given
  \[e_*(e_{i+1}\otimes dx\otimes dx = e_i\otimes d(x\otimes x)\]
\end{Lemma}

\begin{proof}
  The first part is easy, the map compresses because relevant quotient is higher dimensional than $W^{i+1}\ltimes\Gamma_{j+1}$. The second part follows from the lemma that the source and target elements are homologous mod 2.  
\end{proof}




\subsection{Reduction to Three Cases}

We can now attempt to reduce differentials in the adams spectral sequence to 3 cases.  
Let $x\in E_r^{s,s+n}(S,Y)$ and consider $Sq^jx$, which we wish to describe in terms of $d_rx$.  
Let $k=j-n$.  
Represent $x$ by a map
\[(e^n,S^{n-1})\to (Y_s,Y_s+r)\]
For some unknown reason, let $D^k\Gamma_i$ be the $2n-k+i$ skeleton of $W\ltimes \Gamma_i$.  
Let $\xi$ denote maps from $D^k\Gamma_i$ to $Y_{2s+ir-k}$.  

The following diagram commutes
\begin{diagram}
  D^{k+1}\Gamma_2 & \rTo^e & D^k\Gamma_1\\
  \dTo^\xi & & \dTo^\xi \\
  & & Y_{2s+r-k}\\
  & & \dTo\\
  Y_{2(s+r)-k-1} & \rTo & Y_{2s+r-k-1}
\end{diagram}


 

