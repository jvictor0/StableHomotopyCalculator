\documentclass[a4paper,12pt]{article}
\usepackage{enumerate}
\usepackage{graphicx}
\usepackage{amsmath}
\usepackage{amsthm}
\usepackage{amsfonts}
\usepackage[small,nohug,heads=vee]{diagrams}
\diagramstyle[labelstyle=\scriptstyle]


\begin{document}

\newcommand{\A}[0]{\mathcal{A}}
\newcommand{\W}[0]{\mathcal{W}}
\newcommand{\U}[0]{\mathcal{U}}
\newcommand{\Z}[0]{\mathbb{Z}}

\newcommand{\mapsfrom}[0]{\mbox{\reflectbox{$\mapsto$}}}
\newcommand{\from}[0]{\mbox{\reflectbox{$\to$}}}


\newtheorem{Def}{Definition}
\newtheorem{Theorem}{Theorem}
\newtheorem{Lemma}{Lemma}
\newtheorem{Unknown}{Unknown}


\title{Notes on $H_{\infty}$ Ring Spectra For $E_r^{*,*}(S^0,S^0)$ for $p=2$}
\author{Joseph Victor}
\maketitle

The point of this writeup is to help me understand fully the contents of the book $H_\infty$ Ring Spectra by May, Bruner, McClur and Steinberger, especially as it pertains to my situation.
I'm going to try to do all the deails as much as possible, and include details that mean my computations equal the ones they do.  
I will use the tag Unknown to mark unknown things, for instance, like:

\begin{Unknown}
  Why is this thing actually an injective resolution
\end{Unknown}

I will mostly be dealing with chapters that Bruner writes.  
I hope to fully understand all of Bruner's differentials.  

Before we start, some known unknowns:

\begin{Unknown}
  What is the true significance of the symbol 
  \[D_2(X)=\frac{S^\infty\times_{\Z/2} X\wedge X}{S^\infty\times_{\Z/2}\{*\}}\]
\end{Unknown}

\begin{Unknown}
  Why does Bruner use injective resolutions for everything?
  Is it just so that you can use any $E$?  
  Certainly all the theorems true over arbitrary $E$ are true over $H\Z/2$, are the proofs easier over $E$?
\end{Unknown}
  

\section{Homotopy Theory of $H_\infty$ Ring Spectra}

Let's set up some theory so that the differentials can start falling

\subsection{Cohomology of Hopf Algebras}

Let $A$ be a $k$-Hopf Algebra.  We have then
\[\psi:A\to A\otimes A\]
\[\eta:k\to A\]
\[\epsilon:A\to k\]

They satisfy some obvious diagrams.  

\begin{Def}
  A right $A$-comodule is what you think it is.  It should be a $k$-module that has
  \[\psi_M:M\to M\otimes A\]
  Some diagrams commute, just like for Hopf Algebras
\end{Def}

The category of comodules is abelian if $A$ is $k$-flat, which is is.  
We have $Hom_A(M,N)\subseteq Hom_k(M,N)$ for $A$-comodules $M,N$.  

There is a functor adjoint to forgetting the comodule structure that goes
\[P\leadsto P\otimes A\]
\begin{Def}
  We call such a comodule \emph{extended}
\end{Def}



Of course we have
\[Hom_k(M,P)\cong Hom_A(M,p\otimes A)\]
where $M$ is any comodule, but on the left side it is considered just as a $k$-module.

by sending
\[f\mapsto (f\otimes1)\circ \psi_M\]
\[(1\otimes\epsilon)\circ f \mapsfrom f\]

\begin{Lemma}
  Retracts of extended comodules form an injective class relative to the R-split exact sequences, and we have a comparison theorem.
  In particular this applies to direct sums and extended comodules themselves.  
\end{Lemma}

\begin{proof}
  This seems a bit strange at first, but let $f:X\to Y$ be an injection and $g:X\to P\otimes A$.  
  Also, let $\sigma:Y\to X$ be $k$-section to $f$.  Then, letting $\hat{g}=(1\otimes \epsilon)g$, we have
  \[h=(\hat{g}\sigma\otimes 1)\psi_Y\]
  extends $g$ to $Y$, since we are just using adjoints
\end{proof}

For a $k$-split exact sequence, we may choose the splittings to be differentials.  

We can have a diagonal coproduct on $M\otimes A$ as well as the extended one.  They are isomorphic by the adjoint of $1\otimes \epsilon$. 

Let $\overline{A}$ be the cokernel of $\eta$, $p$ the projection and $t$ the $k$-splitting such that $tp=1-\eta\epsilon$.  
If $M$ is an $A$-comodule then we have a $k$-split short exact sequence of $A$-comodules 
\[0\to M\xrightarrow{1\otimes\eta} M\otimes A\xrightarrow{1\otimes p} M\otimes \hat{A}\to 0\]
split by $1\otimes \epsilon$ and $1\otimes t$.  

We get the following cannonical injective resolution

\begin{Def}
  The \emph{normalized canonical resolution} $C(A,M)$ is $C_*$ where
  \[C_s=M\otimes \overline{A}^s\otimes A\]
  \[d_s=(1\otimes\eta)(1\otimes p)\]
  and the splitting $\sigma_s$ is
  \[\sigma_s=(1\otimes t)(1\otimes \epsilon)\]
  We say the homological degree is $s$, the internal degree is $t=$ the sume of the degrees of the tensor factors and total degree is $t-s$.  
  Define also $C(N,A,M)$ by 
  \[C_{s,t}(N,A,M)=Hom_A^t(N,C_s(A,M))\]
\end{Def}

\begin{Theorem}
  $C(A,M)$ is a an injective resolution of $M$
\end{Theorem}

\begin{proof}
  Obviously we're good on the injective part, since its split and $d^2=\sigma^2=0$.  
  $\sigma$ is a contracting homomorphism, so this is also exact.
\end{proof}

\subsection{Products and Steenrod Operations in Ext}

Let $M,N$ be $A$ comodules such that $M$ is an $A$-algebra and $N$ is an $A$-coalgebra.
Write $\phi:M^r\to M$ and $\Delta:N\to M^r$ to be iterated products.  
We write $\epsilon_N:N\to k$ and $\eta_M:k\to M$ as the unit and counit.  
We have a unit 
\[Ext_A(k,k)\to Ext_A(N,M)\]
given by, if $u$ is a cocycle in $C(k,A,k)$, then we have $(\epsilon_N\otimes 1)u\eta_N$ is a cocycle in $C(N,A,M)$.  

We can of course form $C(A,M)\otimes C(A',M')$ as an $A\otimes A'$-comodule with contracting homotopy $\sigma\otimes1+\eta\epsilon\otimes\sigma$.  
There is, by the comparison theorem, a unique chain homotopy class of $A\otimes A'$-maps between this and $C(A\otimes A',M\otimes M')$ over the identity.
Using the coproduct we can make $C^n$ into an $A-comodule$ for any $A$-comodule $A$, and of course we have a map
\[C(A,M)^n\to C(A,M)\]
over the product $\phi$.  This map makes $C(A,M)$ into a DGA.  The functoriality of the tensor product and $Hom(\Delta,\phi)$ makes $C(N,A,M)$ into a $k$-DGA, so $Ext_A(N,M)$ is an algebra over $Ext_A(R,R)$.  

We now want to make the Steenrod Operations work out, so let $k=\mathbb{F}_2$ and $R=k[\alpha]/(\alpha^2)$.  
Let $\W_*$ be the cannonical resolution of $R$ as a a group ring, and $e_i$ generates $\W_i$.
Let $K$ be a chain complex, and let $Z/2$ act on $K^2$ by flipping and diagonally on $\W\otimes K^2$.  
Assume further that $K$ is a $k$-DGA and we have an $R$ map
\[\theta:\W\otimes K^2\to K\]
with
\[\theta(e_0\otimes x\otimes y) = xy\]
We also assme
\[\mu K\otimes K\to K\]
satisfies
\[\mu(\theta\otimes\theta)\sim \theta(1\otimes\mu\otimes\mu)\]
Letting $\U$ be any $k[S_4]$-free resolution of $k$, and let $\tau=Z/4\subset S_4$.  
Let 
\[\omega:\W\otimes \W\otimes \W\to \U\]
be a $k[\tau]$ chain map over the identity.
Can we make the following diagram commute (up to homotopy) with a $\xi$?
\begin{diagram}
  \W\otimes\W\otimes \W\otimes K^4 & \rTo^{\omega\otimes 1} & \U\otimes K^4  \\
  \dTo &                              &                           &           \rdTo^\xi\\                                   
     &                                &                            &           &    K \\
   &                              &                           &           \ruTo^\theta\\                                   
  \W\otimes \W\otimes K^2 \otimes \W\otimes K^2 & \rTo^{1\otimes \theta\otimes \theta} & \W\otimes K^2
\end{diagram}

\begin{Unknown}
\label{ademdiag}
Why do we care?
I think the idea might be that this makes the Adem relation work out, for some reason.  
When I figure this out I might move this diagram to after the steenrod ops are defined.  
I should probably also figure out if my definition is equivalent.  
The way to do that is of course to say that they satisfy the same properties which uniquely define the operations.
\end{Unknown}

We show for $K-C(A,M)$ that $\theta$ and $\xi$ exist.  

\begin{Lemma}
  Let $G=Z/r$ be a cyclic group and $K=C(A,M)$.  
  let $\U$ be a free $k[G]$ resolution of $k$.  
  Then there is a unique $G$-equivarient map
  \[\Phi:\U\otimes K^r\to K\]
  such that
  \[\Phi(\U_0\otimes k_1\otimes ... \otimes k_r)=k_1...k_r\]
  and if $ri>(r-1)j$ then 
  \[\Phi(\U_i\otimes (K^r)_j)=0\]
\end{Lemma}

\begin{proof}
  We write $\Phi_{i,j}=\Phi|\U_i\otimes (K^r)_j$.
  We automatically get $\Phi_{0,*}$.  
  We let $\tilde{\Phi}_{i,j}$ be the adjoint of $\Phi$, and $S$ be the contracting homotopy of $K^r$.  
  Define
  \[\tilde\Phi_{i,j}=(d\tilde\Phi_{i,j-1}-\tilde{\Phi}_{i-1.j-1}\widetilde{(d\otimes 1)})(1\otimes S)\]
  A lengthy but simple induction shows this is a chain map, 0 when it should be.  \cite[Page 98]{hrs}

  We now show uniqueness.  Let $\theta$ also satisfy the conditions, but one can simply use adjoints and contracting homotopies to define a homotopy inductively.  
\end{proof}

We can now define 
\[\Phi_* : \W\otimes C(N,A,M)^2\to C(N,A,M)\]
by
\[\Phi_*(w\otimes \varphi_1\otimes \varphi_2)=n\mapsto \Phi(w\otimes((\varphi_1\otimes\varphi_2)\Delta(n)))\]


\begin{Def}
  Let $x\in Ext^{s,t}_A(N,M)$.  Define
  \[Sq^i(x)=\Phi_*(e_{i-t+s}\otimes x\otimes x)\]
\end{Def}

\begin{Theorem}
  We have natrual maps:

  \begin{enumerate}
    \item $Sq^i :Ext_A^{s,t}\to Ext_A^{s+t-2i,2t}$ 
    \item $Sq^i=0$ unless $t_s\le i \le t$.  
    \item $Sq^{t-s}(x)=x^2$
    \item The Cartan Formula holds
    \item The Adem relation holds
  \end{enumerate}
\end{Theorem}

\begin{proof}
  The first three are easy.  $C(N,A,M)$ is something called a Cartan Object and an Adem Object, implying the other two
\end{proof}

\begin{Unknown}
  What on earth is a Cartan object or Adem object, and what implies this?
  Apparently the diagram I said was satisfied by $\xi$ implies the Adem relation, while somehow co-commutativity implies Cartan.  
\end{Unknown}

\subsection{The Adams Spectral Sequence}

Let
\[Y_0\from Y_1\from Y_2\from ...\]
be a sequence of spaces and let $Y_{s,r}=Y_s/Y_{s+r}$.  Then we have fibrations
\begin{diagram}
  Y_{s+r}& &\rTo^i & & Y_s \\
  &\luTo^\partial & & \ldTo^p\\
  && Y_{s,r}
\end{diagram}
which induce the exact couple 
\begin{diagram}
  \bigoplus [X,Y_{s}]_{t-s}& &\rTo^{i_*} & & \bigoplus [X,Y_s]_{t-s} \\
  &\luTo^{\partial_*} & & \ldTo^{p_*}\\
  && \bigoplus[X,Y_{s,1}]_{t-s}
\end{diagram}

I enjoy writing it as a staircase diagram, making some lemmas easier to visualize.  

\begin{diagram}
  \vdots &&&& \vdots  &&&& \vdots \\
  \dTo & &&& \dTo  & &&& \dTo &&& &  \\
  [X,Y_s]_{t-s+1} & \rTo & [X,Y_{s,1}]_{t-s+1} & \rTo & [X,Y_{s+1}]_{t-s} & \rTo & [X,Y_{s+1,1}]_{t-s} & \rTo & [X,Y_{s+2}]_{t-s-1}\\
  \dTo & &&& \dTo  & &&& \dTo &&& &  \\
  [X,Y_{s-1}]_{t-s+1} & \rTo & [X,Y_{s-1,1}]_{t-s+1} & \rTo & [X,Y_{s}]_{t-s} & \rTo & [X,Y_{s,1}]_{t-s} & \rTo & [X,Y_{s+1}]_{t-s-1}\\
  \dTo & &&& \dTo  & &&& \dTo &&& &  \\
  [X,Y_{s-2}]_{t-s+1} & \rTo & [X,Y_{s-2,1}]_{t-s+1} & \rTo & [X,Y_{s-1}]_{t-s} & \rTo & [X,Y_{s-1,1}]_{t-s} & \rTo & [X,Y_{s}]_{t-s-1}\\
  \dTo & &&& \dTo  & &&& \dTo &&& &  \\
  \vdots &&&& \vdots  &&&& \vdots 
\end{diagram}


Of course the differentials can be calculated as follows: if $f\in [X,Y_{s,1}]_{t-s}$, then it has some image in $\overline{f}\in [X,Y_{s+1}]_{t-s+1}$.  
If the image of $\overline{f}$ can be homotopicly restricted to $Y_{s+r}$, then $f$ survives to $E_r$ and its image in $[X,Y_{s+r}]_{t-s+1}$ represents $d_r(f)$, which is a permanent cycle.  

\begin{Lemma}
  
\end{Lemma}



\begin{thebibliography}{99}

\bibitem{adams1}
  Frank Adams
  \emph{On Structure and Applications of the Steenrod Algebra}
  
\bibitem{ElienCart}
  Samuel Eilenberg and Henri Cartan
  \emph{Homological Algebra}

\bibitem{MasseyEC}
  W.S. Massey
  \emph{Products In Exact Couples}

\bibitem{Kahn1}
  Daniel Kahn
  \emph{Cup-i Products and the Adams Spectral Sequence}

\bibitem{luiCycs}
  TODO: Fill out this entry

\bibitem{hrs}
  TODO: Fill out thsi very important entry

\end{thebibliography}
\end{document}
