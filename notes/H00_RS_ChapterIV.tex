\section{Homotopy Theory of $H_\infty$ Ring Spectra}

Let's set up some theory so that the differentials can start falling

\subsection{Cohomology of Hopf Algebras}

Let $A$ be a $k$-Hopf Algebra.  We have then
\[\psi:A\to A\otimes A\]
\[\eta:k\to A\]
\[\epsilon:A\to k\]

They satisfy some obvious diagrams.  

\begin{Def}
  A right $A$-comodule is what you think it is.  It should be a $k$-module that has
  \[\psi_M:M\to M\otimes A\]
  Some diagrams commute, just like for Hopf Algebras
\end{Def}

The category of comodules is abelian if $A$ is $k$-flat, which is is.  
We have $Hom_A(M,N)\subseteq Hom_k(M,N)$ for $A$-comodules $M,N$.  

There is a functor adjoint to forgetting the comodule structure that goes
\[P\leadsto P\otimes A\]
\begin{Def}
  We call such a comodule \emph{extended}
\end{Def}



Of course we have
\[Hom_k(M,P)\cong Hom_A(M,p\otimes A)\]
where $M$ is any comodule, but on the left side it is considered just as a $k$-module.

by sending
\[f\mapsto (f\otimes1)\circ \psi_M\]
\[(1\otimes\epsilon)\circ f \mapsfrom f\]

\begin{Lemma}
  Retracts of extended comodules form an injective class relative to the R-split exact sequences, and we have a comparison theorem.
  In particular this applies to direct sums and extended comodules themselves.  
\end{Lemma}

\begin{proof}
  This seems a bit strange at first, but let $f:X\to Y$ be an injection and $g:X\to P\otimes A$.  
  Also, let $\sigma:Y\to X$ be $k$-section to $f$.  Then, letting $\hat{g}=(1\otimes \epsilon)g$, we have
  \[h=(\hat{g}\sigma\otimes 1)\psi_Y\]
  extends $g$ to $Y$, since we are just using adjoints
\end{proof}

For a $k$-split exact sequence, we may choose the splittings to be differentials.  

We can have a diagonal coproduct on $M\otimes A$ as well as the extended one.  They are isomorphic by the adjoint of $1\otimes \epsilon$. 

Let $\overline{A}$ be the cokernel of $\eta$, $p$ the projection and $t$ the $k$-splitting such that $tp=1-\eta\epsilon$.  
If $M$ is an $A$-comodule then we have a $k$-split short exact sequence of $A$-comodules 
\[0\to M\xrightarrow{1\otimes\eta} M\otimes A\xrightarrow{1\otimes p} M\otimes \hat{A}\to 0\]
split by $1\otimes \epsilon$ and $1\otimes t$.  

We get the following cannonical injective resolution

\begin{Def}
  The \emph{normalized canonical resolution} $C(A,M)$ is $C_*$ where
  \[C_s=M\otimes \overline{A}^s\otimes A\]
  \[d_s=(1\otimes\eta)(1\otimes p)\]
  and the splitting $\sigma_s$ is
  \[\sigma_s=(1\otimes t)(1\otimes \epsilon)\]
  We say the homological degree is $s$, the internal degree is $t=$ the sume of the degrees of the tensor factors and total degree is $t-s$.  
  Define also $C(N,A,M)$ by 
  \[C_{s,t}(N,A,M)=Hom_A^t(N,C_s(A,M))\]
\end{Def}

\begin{Theorem}
  $C(A,M)$ is a an injective resolution of $M$
\end{Theorem}

\begin{proof}
  Obviously we're good on the injective part, since its split and $d^2=\sigma^2=0$.  
  $\sigma$ is a contracting homomorphism, so this is also exact.
\end{proof}

\subsection{Products and Steenrod Operations in Ext}

Let $M,N$ be $A$ comodules such that $M$ is an $A$-algebra and $N$ is an $A$-coalgebra.
Write $\phi:M^r\to M$ and $\Delta:N\to M^r$ to be iterated products.  
We write $\epsilon_N:N\to k$ and $\eta_M:k\to M$ as the unit and counit.  
We have a unit 
\[Ext_A(k,k)\to Ext_A(N,M)\]
given by, if $u$ is a cocycle in $C(k,A,k)$, then we have $(\epsilon_N\otimes 1)u\eta_N$ is a cocycle in $C(N,A,M)$.  

We can of course form $C(A,M)\otimes C(A',M')$ as an $A\otimes A'$-comodule with contracting homotopy $\sigma\otimes1+\eta\epsilon\otimes\sigma$.  
There is, by the comparison theorem, a unique chain homotopy class of $A\otimes A'$-maps between this and $C(A\otimes A',M\otimes M')$ over the identity.
Using the coproduct we can make $C^n$ into an $A-comodule$ for any $A$-comodule $A$, and of course we have a map
\[C(A,M)^n\to C(A,M)\]
over the product $\phi$.  This map makes $C(A,M)$ into a DGA.  The functoriality of the tensor product and $Hom(\Delta,\phi)$ makes $C(N,A,M)$ into a $k$-DGA, so $Ext_A(N,M)$ is an algebra over $Ext_A(R,R)$.  

We now want to make the Steenrod Operations work out, so let $k=\mathbb{F}_2$ and $R=k[\alpha]/(\alpha^2)$.  
Let $\W_*$ be the cannonical resolution of $R$ as a a group ring, and $e_i$ generates $\W_i$.
Let $K$ be a chain complex, and let $Z/2$ act on $K^2$ by flipping and diagonally on $\W\otimes K^2$.  
Assume further that $K$ is a $k$-DGA and we have an $R$ map
\[\theta:\W\otimes K^2\to K\]
with
\[\theta(e_0\otimes x\otimes y) = xy\]
We also assme
\[\mu K\otimes K\to K\]
satisfies
\[\mu(\theta\otimes\theta)\sim \theta(1\otimes\mu\otimes\mu)\]
Letting $\U$ be any $k[S_4]$-free resolution of $k$, and let $\tau=Z/4\subset S_4$.  
Let 
\[\omega:\W\otimes \W\otimes \W\to \U\]
be a $k[\tau]$ chain map over the identity.
Can we make the following diagram commute (up to homotopy) with a $\xi$?
\begin{diagram}
  \W\otimes\W\otimes \W\otimes K^4 & \rTo^{\omega\otimes 1} & \U\otimes K^4  \\
  \dTo &                              &                           &           \rdTo^\xi\\                                   
     &                                &                            &           &    K \\
   &                              &                           &           \ruTo^\theta\\                                   
  \W\otimes \W\otimes K^2 \otimes \W\otimes K^2 & \rTo^{1\otimes \theta\otimes \theta} & \W\otimes K^2
\end{diagram}

\begin{Unknown}
\label{ademdiag}
Why do we care?
I think the idea might be that this makes the Adem relation work out, for some reason.  
When I figure this out I might move this diagram to after the steenrod ops are defined.  
I should probably also figure out if my definition is equivalent.  
The way to do that is of course to say that they satisfy the same properties which uniquely define the operations.
\end{Unknown}

We show for $K-C(A,M)$ that $\theta$ and $\xi$ exist.  

\begin{Lemma}
  Let $G=Z/r$ be a cyclic group and $K=C(A,M)$.  
  let $\U$ be a free $k[G]$ resolution of $k$.  
  Then there is a unique $G$-equivarient map
  \[\Phi:\U\otimes K^r\to K\]
  such that
  \[\Phi(\U_0\otimes k_1\otimes ... \otimes k_r)=k_1...k_r\]
  and if $ri>(r-1)j$ then 
  \[\Phi(\U_i\otimes (K^r)_j)=0\]
\end{Lemma}

\begin{proof}
  We write $\Phi_{i,j}=\Phi|\U_i\otimes (K^r)_j$.
  We automatically get $\Phi_{0,*}$.  
  We let $\tilde{\Phi}_{i,j}$ be the adjoint of $\Phi$, and $S$ be the contracting homotopy of $K^r$.  
  Define
  \[\tilde\Phi_{i,j}=(d\tilde\Phi_{i,j-1}-\tilde{\Phi}_{i-1.j-1}\widetilde{(d\otimes 1)})(1\otimes S)\]
  A lengthy but simple induction shows this is a chain map, 0 when it should be.  \cite[Page 98]{hrs}

  We now show uniqueness.  Let $\theta$ also satisfy the conditions, but one can simply use adjoints and contracting homotopies to define a homotopy inductively.  
\end{proof}

We can now define 
\[\Phi_* : \W\otimes C(N,A,M)^2\to C(N,A,M)\]
by
\[\Phi_*(w\otimes \varphi_1\otimes \varphi_2)=n\mapsto \Phi(w\otimes((\varphi_1\otimes\varphi_2)\Delta(n)))\]


\begin{Def}
  Let $x\in Ext^{s,t}_A(N,M)$.  Define
  \[Sq^i(x)=\Phi_*(e_{i-t+s}\otimes x\otimes x)\]
\end{Def}

\begin{Theorem}
  We have natrual maps:

  \begin{enumerate}
    \item $Sq^i :Ext_A^{s,t}\to Ext_A^{s+t-2i,2t}$ 
    \item $Sq^i=0$ unless $t_s\le i \le t$.  
    \item $Sq^{t-s}(x)=x^2$
    \item The Cartan Formula holds
    \item The Adem relation holds
  \end{enumerate}
\end{Theorem}

\begin{proof}
  The first three are easy.  $C(N,A,M)$ is something called a Cartan Object and an Adem Object, implying the other two
\end{proof}

\begin{Unknown}
  What on earth is a Cartan object or Adem object, and what implies this?
  Apparently the diagram I said was satisfied by $\xi$ implies the Adem relation, while somehow co-commutativity implies Cartan.  
\end{Unknown}

\subsection{The Adams Spectral Sequence}

Let
\[Y_0\from Y_1\from Y_2\from ...\]
be a sequence of spaces and let $Y_{s,r}=Y_s/Y_{s+r}$.  Then we have fibrations
\begin{diagram}
  Y_{s+r}& &\rTo^i & & Y_s \\
  &\luTo^\partial & & \ldTo^p\\
  && Y_{s,r}
\end{diagram}
which induce the exact couple 
\begin{diagram}
  \bigoplus [X,Y_{s}]_{t-s}& &\rTo^{i_*} & & \bigoplus [X,Y_s]_{t-s} \\
  &\luTo^{\partial_*} & & \ldTo^{p_*}\\
  && \bigoplus[X,Y_{s,1}]_{t-s}
\end{diagram}

I enjoy writing it as a staircase diagram, making some lemmas easier to visualize.  

\begin{diagram}
  \vdots &&&& \vdots  &&&& \vdots \\
  \dTo & &&& \dTo  & &&& \dTo &&& &  \\
  [Y,X_s]_{t-s+1} & \rTo & [Y,K_{s}]_{t-s+1} & \rTo & [Y,X_{s+1}]_{t-s} & \rTo & [Y,K_{s+1}]_{t-s} & \rTo & [Y,X_{s+2}]_{t-s-1}\\
  \dTo & &&& \dTo  & &&& \dTo &&& &  \\
  [Y,X_{s-1}]_{t-s+1} & \rTo & [Y,K_{s-1}]_{t-s+1} & \rTo & [Y,X_{s}]_{t-s} & \rTo & [Y,K_{s}]_{t-s} & \rTo & [Y,X_{s+1}]_{t-s-1}\\
  \dTo & &&& \dTo  & &&& \dTo &&& &  \\
  [Y,X_{s-2}]_{t-s+1} & \rTo & [Y,K_{s-2}]_{t-s+1} & \rTo & [Y,X_{s-1}]_{t-s} & \rTo & [Y,K_{s-1}]_{t-s} & \rTo & [Y,X_{s}]_{t-s-1}\\
  \dTo & &&& \dTo  & &&& \dTo &&& &  \\
  \vdots &&&& \vdots  &&&& \vdots 
\end{diagram}


Of course the differentials can be calculated as follows: if $f\in [X,Y_{s,1}]_{t-s}$, then it has some image in $\overline{f}\in [X,Y_{s+1}]_{t-s+1}$.  
If the image of $\overline{f}$ can be homotopicly restricted to $Y_{s+r}$, then $f$ survives to $E_r$ and its image in $[X,Y_{s+r}]_{t-s+1}$ represents $d_r(f)$, which is a permanent cycle.  

\begin{Lemma}
  \cite[3.3,3.4,pate 103]{hrs}
\end{Lemma}

Let $H=HZ/2$ and recall $H_*(X)=\pi_*(X\wedge HZ/2)$.  Recall, by the universal coeficient theorem, that $H^*(H)=Hom_k(\A,k)=\A_*$.  
Recall that the dual steenrod algebra is a polynomial algebra on generators $\xi_k$ of degree $2^k-1$.  
The element $\xi_k$ is dual to the admisable sequence $Sq^{\{2^{i-1},2^{i-2},...,Sq^1\}}$.  
The comultiplication is of course given
\[\phi(\xi_k)=\sum_{i=0^k}\xi^{2^i}_{k-i}\otimes \xi_i\]

We use the following Adams resolution
\begin{Def}
  The Adams resolution of a space $Y$ is a sequence $Y_0=Y$, $Y_{s+1}=Y_s\wedge\overline{H}$.  
  Notice that this makes $Y_{s,1}=Y_s\wedge H$
\end{Def}


Since $H_*(Y_{s,1})=H_*(Y_s\wedge H)=H_*(Y_s)\otimes \A_*$ is an extended $\A_*$ comodule, we can splice the $Y_i$'s together to get the following injective resolution
\[0\to H_*(Y)\to H_*(Y_{0,1})\to H_*(\Sigma Y_{1,1})\to H_*(\Sigma^2 Y_{2,1})\to ...\]
This means the $E_2$ page is indeed Ext.


\begin{Lemma}
  This is the cobar resolution.  The $E_1$ term of the resulting spectral sequence is $C(H_*X, \A_*,H_*Y)$.  
\end{Lemma}

This follows from the fact that $\overline{A_*}=H_*(\overline{H})$ and the Kunneth isomorphism, which I've been using no problem.  
We have, also that 
\[[X,Y_{s,1}]_*=Hom_{\A_*}(H_*X,H_*Y_{s,1})\]
so the $E_2$ term is what we want it to be.  


\begin{Lemma}
  If $f:X\to Y$ is extended to a chain map of $H_*(X_{*,1})\to H_*(Y_{*,1})$, then we can extend $f$ to a map of Adam's resolutions.
\end{Lemma}

\begin{proof}
  Use the 5 lemma of spectra and the isomorphism $[Z,Y_{i,1}]_*=Hom_{\A_*}(H_*(Z),H_*(Y_{i,1}))$
\end{proof}

\subsection{Smash Products in the Adams Spectral Sequence}

\begin{Theorem}
Smashing representative maps give a product
\[E_r^{*,*}(X,Y)\otimes E_r^{*,*}(X',Y') \to E_r^{*,*}(X\wedge X',Y\wedge Y')\]
\end{Theorem}
Since the smash product of Adams resolutions is an Adams resolution, this is easy to show.
We also automatically have DGA structure for $E_*^{*,*}(S,S)$.  
For other $E\ne H$, this is actually hard, but we don't worry about that here.

\subsection{Extended Powers in the Adams Spectral Sequence}

Recall
\[D_2Y=S^2 \ltimes_{\Z/2} (Y\wedge Y)\]
the wedge product has the obvious $Z/2$ action and $S^2$ has the antipodal map.  

Suppose given a map
\[\xi:D_2Y \to Y\]
extending $Y\wedge Y\to Y$.  We want to use this to make steenrod operations explicit in $Ext_{\A_*}(M,H_*Y)$.  

Let $W_i$ be the $i$-skeleton of a $\Z/2$-free CW complex $W$ which is contractible, and $W_0=\Z/2$.  Define
\[D_\pi^iY=W_i\ltimes_{\Z/2} (Y\wedge Y)\]
 
Let $Y_i$ be an Adams resolution.  Let $F_*=Y_*\wedge Y_*$.  Define $Z_{i,s}=W_i\ltimes_{\Z/2} F_s$ and $B_i=W_i/\Z/2$. 

\begin{Lemma}
  We have some facts about spaces:
  \begin{enumerate}
    \item $Z_{i-1,s}$ and $Z_{i,s+1}$ are subcomplexes of $Z_{i,s}$.  
    \item $\frac{Z_{i,s}}{Z_{i-1,s}}=\frac{B_i}{B_{i-1}}\wedge F_s$
    \item $\frac{Z_{i,s}}{Z_{i-1,s}\cup Z_{i,s+1}}=\frac{B_i}{B_{i-1}}\wedge \frac{F_s}{F_{s+1}}$
    \item the diagram on page 111 commutes
  \end{enumerate}
\end{Lemma}

\begin{proof}
  maybe later
\end{proof}

\begin{Theorem}
  There exist maps $\xi_{i,s}:Z_{i,s}\to Y_{s-1}$ with the following diagram commuting.  
\[  \begin{diagram}
    D_2Y & \lTo & Z_{i,s}\\
    \dTo^\xi & & \dTo^{\xi_{i,s}}\\
    Y & \lTo & Y_{s-i}
  \end{diagram}\hspace{15mm}
  \begin{diagram}
    Z_{i,s-1} & \lTo & Z_{i,s} & \lTo & Z_{i-1,s}\\
    \dTo^{\xi_{i,s-1}} & & \dTo^{\xi_{i,s}} & &\dTo^{\xi_{i-1,s}}\\
    Y_{s-1-i} & \lTo & Y_{s-i} & \lTo & Y _{s-i+1}
  \end{diagram}\]
\end{Theorem}


\begin{proof}
  When $i=0$ this is just the extension of the smash product $F_s\to Y_s$.  
  By induction suppose we need to define $\xi_{k,s}$.  
  The obstruction lies in $[Z_{k,s}/Z_{k-1,s},Y_{s-k-1,1}]$.
  This is of course equal to a $Hom_{\A_*}$ group, and the inclusion of $F_s\to F_{s-1}$ induces 0 in homology, the obstruction is 0.  
\end{proof}

Let $\W_k=\pi_k(W_k/W_{k-1})$, $h$ be Hurewics and $\kappa$ be Kunneth.  
Then $\Phi=(\xi_{k,s})_*\kappa(h\otimes 1)$.   This means we can get an explicit class of $Sq^i$.   

\subsection{Milgram's Generalization}

\begin{Unknown}
  What is the point of this Spectral Sequence
\end{Unknown}

\begin{Theorem}
  Let 
  \[\z=Z_0\from^{f_0} Z_1\from^{f_1} Z_2\from ...\]
  then
  \begin{enumerate}
    \item there is a SS 
      \[E_2^{s,t}(X,\z)=\bigoplus E_2^{s-i,t-i}(X,Cf_i)\]
      which converges to $[X,Cf_i]_*$
    \item It has a smash product pairing
    \item A map of $\z$ to the smash product pairing gives a map of spectral sequences converging to a map of homotopy groups
    \item The smash product pairing agrees with such induced maps
  \end{enumerate}
\end{Theorem}

We omit the proof, since it follows pretty much the same logic as the original and is very technical.  

\subsection{Homotopy operations for $H_\infty$ Ring Spectra}

\begin{Unknown}
  Why if $k=1$ is the only useful, why state for general $k$
\end{Unknown}

\begin{Def}
  Let $\alpha\in Y_m(D_jS^n)$.  Then define 
  \[\alpha^*:\pi_n\to \pi_m\]
  by the composite 
  \[S^m\xrightarrow{\alpha} D_jS^n\wedge Y\xrightarrow{D_jf\wedge 1} D_jY\xrightarrow{\xi} Y\]
\end{Def}

\begin{Lemma}
  \begin{enumerate}
    \item $\alpha^*$ is natrual
    \item $?^*$ is additive
    \item The inclusion $i:S^{2n}\to D_2S^n$ induces $i^*(x)=x^2$
  \end{enumerate}
\end{Lemma}

We adopt the notation $x\in \pi_nY$ is detected by $\overline{x}\in E_2^{s,n+s}(S,Y)$.  

We have a sequence 
\[D=D_2^{2s}S^n\from ...\from D_2^0S^{2n}\]  

We have for each $i$
\[D^i_2S^n\xrightarrow{D_2(x)} D^i_2Y_S\xrightarrow{\xi_{i,2s}} Y_{2s-i}\]
and so by section 6 we have
\[P(x):E^{*,*}_r(S,D)\to E^{*,*}_r(S,Y)\]
and
\[P(x):E^{*,*}_r(S,D\wedge Y)\to E^{*,*}_r(S,Y)\]

\begin{Lemma}
  $E_2^{*,*}(S,D)$ is free over $E_2(S,S)$ on enerators $e_i\in E_2^{2s-i,2s+2n}(S,D)$
\end{Lemma}

This follows from the deffinition of the spectral sequence as a direct sum.  
Note that $e_i$ is the $2n_i$-cell of $D_2S^n$.

\begin{Theorem}
  $P(x)(e_i)=Sq^{i+n}(\overline{x})$
\end{Theorem}

\begin{proof}
  We have $P(x)(e_i)$ is represented by
  \[S^{2n+i}\xrightarrow{e_i} \frac{W_i}{W_{i-1}}\wedge S^n\wedge S^n\xrightarrow{1\wedge \overline{x}\wedge \overline{x}}\frac{W_i}{W_{i-1}}\wedge Y_{s,1}\wedge Y_{s,1}
  \xrightarrow{\xi_{i,2s}} Y_{2s-i,1}\]
\end{proof}

Since this converges, we have $\alpha\in \pi_XD^{2s}_2S^n$ is detected by some $\sum a_ke_k$ with $a_k\in E_2(S,S)$.  
We have then that $\alpha^*(x)$ is detected by $\sum a_kSq^{k+n}(\overline{x})$.  The same holds for $Y_*D_2^{2s}S^n$.  

\begin{Lemma}
  if $d_r(ae_k)=\sum a_ie_{k_i}$ in $E_r(S,D\wedge Y)$ then 
  \[d_r(aSq^{n+k}\overline{x})=\sum a_iSq^{k_i+n}\overline{x}\]
\end{Lemma}

With the exception of the results I skipped, we are now ready for chapter V and real computations.

